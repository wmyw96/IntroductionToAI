\documentclass{article}

\usepackage{graphicx}
\usepackage{xeCJK}
\usepackage{bm}
\usepackage{amsmath,amsthm,amssymb,amsfonts}
\usepackage{color}
\usepackage{geometry}
\geometry{left=3.2cm,right=3.2cm,top=3.2cm,bottom=3.2cm}

%给文字加颜色,用法:
%{\color{red}{I love you}}

%\usepackage{ntheorem}
\newtheorem{them}{定理}[subsection]
\newtheorem{defn}{定义}[subsection]
\newtheorem{lemm}{引理}[subsection]

%\setCJKmainfont[BoldFont = 黑体]{宋体}
%\setlength{\parindent}{2em}
%如果不要缩进 用\noindent
\title{Pinyin Input Method Editor\\Design Report}

\author{\large Yihong Gu\\gyh15@mails.tsinghua.edu.cn\\Department of Computer Science\\Tsinghua University}

\date{}

\begin{document}

\maketitle

\section{Introction}

报告分为三个部分:

\begin{itemize}
	\item Language Model: 介绍所用的语言模型
	\item Search Algorithm: 介绍所用的搜索算法以及优化
	\item Experiments: 给出实验结果并作相关分析
	\item Code Structure: 简要介绍代码结构
\end{itemize}

\section{Language Model}

\subsection{Probability Model}

总体来说,我们使用以下语言模型:

\begin{eqnarray}
	\mathbb{P}(w_1\cdots w_n) = \prod_{i=1}^{\min(n,m)}{\mathbb{P}(w_i\lvert w_{\max(i-m+1,1)} \cdots w_{i-1})}
\end{eqnarray}

我们把这个模型称为m-gram模型。

在这里面$w_i$表示第$i$个汉字,举个例子,取$m=2$:

\begin{eqnarray}
	\mathbb{P}(\text{清华大学})=\mathbb{P}(\text{清})\mathbb{P}(\text{华}\lvert\text{清})\mathbb{P}(\text{大}\lvert\text{华})\mathbb{P}(\text{学}\lvert\text{大})
\end{eqnarray}

事实上,这里面我们没有考虑拼音的影响,那么,我们作最简单的假设,假设拼音和$m-gram$独立并且条件分布是离散分布

\begin{eqnarray}
	\mathbb{P}(w_1\cdots w_n\lvert t_1 \cdots t_n) = \prod_{i=1}^{\min(n,m)}{\mathbb{P}(w_i\lvert w_{\max(i-m+1,1)} \cdots w_i-1)\mathbb{P}(w_i \lvert t_i)}
\end{eqnarray}

我们让

\begin{eqnarray}
	\mathbb{P}(w_i\lvert w_{i-m+1}\cdots w_{i-1}) = \frac{\#\{w_{i-m+1}\cdots w_{m}\}}{\#\{w_{i-m+1}\cdots w_{m-1}\}}
\end{eqnarray}

其中$\#\{w_{i-m+1}\cdots w_{i}\}$为词组$w_{i-m+1}\cdots w_{i}$在corpus中出现的频数,并且让$\mathbb{P}(w_i \lvert t_i)$为$1$当且仅当汉字$w_i$存在发音$t_i$,否则为$0$,我们也尝试了其他的模型(均匀分布,按汉字的词频归一化的离散分布,但是发现实际上这些方法会引入大量噪声,实际效果并没有之前这种简单也不归一化的方法好,因为前一种方法让文本完全由corpus决定,不引入拼音造成的噪声)。

\subsection{Frequency Count}

在计算$\#\{w_{i-m+1}\cdots w_{i}\}$的过程中,我们使用sina新闻2016作为corpus,且把所有的6763个汉字作为$w_i$的字母表$\Sigma$,把新闻正文中不属于$\Sigma$的部分作为分隔符,统计在$\Sigma$中的连续$m$个token(中间不能有分隔符)出现的次数。

由于总的次数过于多,我们考虑只保留部分m-gram的频数统计的结果,我们选取最大的$k$,使得频数$\ge k$的m-gram的频数之和大于总频数之和的$100 \sigma\%$,我们把$\sigma$称为significance level,在这里我们取$\sigma=0.95$,最后我们保留频数$\le k$的m-gram。

\subsection{Probability Smoothing}

首先,为了方便计算,我们同意使用概率取对数进行计算,这样原来的乘积就变成了求和。

由于词频很多时候都为0,所以我们需要用对$\log\mathbb{P}(w_i\lvert w_{i-m+1}\cdots w_{i-1})$进行平滑处理。

我们下面考虑具体的处理过程(递归处理):

\begin{itemize}
	\item 如果当前发现$w_{i-m+1}\cdots w_{i}$和$w_{i-m+1}\cdots w_{i-1}$的频数均非0,那么就直接按照原式计算$\log\mathbb{P}(w_i\lvert w_{i-m+1}\cdots w_{i-1})$。
	\item 如果发现$w_{i-m+1}\cdots w_{i-1}$的频数均为0,并且$m>2$,计算$m'=m-1$的结果$p_{m-1}$,然后输出就是$p_{m-1}-100$,作为平滑处理的惩罚项。
	\item 如果发现$w_{i-m+1}\cdots w_{i-1}$,并且$m=2$,计算$m'=m-1$的结果$p_{m-1}$,然后输出就是$p_{m-1}-2\times 10^8$,作为平滑处理的惩罚项。
\end{itemize}

另外,由于我们是(要通过搜索)需要最大化对数似然值,所以我们设置答案的下界为$-1\times 10^9$,也就是说,像第三项的那种平滑处理不能超过$5$次。

\section{Search Algorithm}

有了Langugae Model后,我们的问题就转变成了最大化

\begin{eqnarray}
	w_1^*\cdots w_n^* = \mathrm{argmax}_{w_1\cdots, w_n} \mathbb{P}(w_1\cdots w_n\lvert t_1 \cdots t_n)
\end{eqnarray}

其中$t_1 \cdots t_n$是给定的拼音,同时$w_1^*\cdots w_n^*$就是我们输出的结果。

我们考虑使用$A^*$算法来解决这个问题。

\subsection{$A^*$ Algorithm}

我们把$w_1\cdots w_i$称为一个状态$s_i$,当$i=n$的时候即到达终点,一个状态$s_i$的收益为$v_i = \log \mathbb{P}(w_1\cdots w_i)$,我们需要最大化到达终点的收益$v_n$。

服从$A^*$的记号,我们发现$g(s_i)=v_i$,另外我们让$h(s_i)=0$,即可用$A^*$来优化。此时我们发现,这个问题实质上变成了一个最长路径问题,这时候的$A^*$也就等价于传统的Dijkstra算法。

\subsection{Improvement}

我们从以下一个角度来优化这个搜索过程:

\noindent \textbf{SLF优化}

我们发现,是否对OPEN表排序(即使用堆来维护OPEN表)不影响时间消耗,所以我们不对OPEN表排序,这样的搜索算法就等价于传统的SPFA算法,我们沿用了SPFA算法的一个非常经典的优化手法$SLF$优化,即如果放入队尾的状态比放入目前队头的要优的话,把队头队尾的元素交换,这样可以使得效率提升3倍。

\noindent \textbf{记忆化}

我们发现,计算local log probability ($\log\mathbb{P}(w_i\lvert w_{i-m+1}\cdots w_{i-1})$)非常消耗时间,所以我们对这一部分进行记忆化,这样效率也可以提升1倍。

政府工作报告7次提及 李克强为何再赠4字?
武汉最懒大学生:两周不收衣服 鸟儿在内做窝
特朗普称叙化武袭击事件是“对人类的羞辱”
郎平:女排备战奥运会培养新人 已着眼下个周期


\end{document}