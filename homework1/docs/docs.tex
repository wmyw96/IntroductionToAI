\documentclass{article}

\usepackage{graphicx}
\usepackage{xeCJK}
\usepackage{bm}
\usepackage{amsmath,amsthm,amssymb,amsfonts}
\usepackage{color}
\usepackage{geometry}
\geometry{left=3.2cm,right=3.2cm,top=3.2cm,bottom=3.2cm}

%给文字加颜色,用法:
%{\color{red}{I love you}}

%\usepackage{ntheorem}
\newtheorem{them}{定理}[subsection]
\newtheorem{defn}{定义}[subsection]
\newtheorem{lemm}{引理}[subsection]

%\setCJKmainfont[BoldFont = 黑体]{宋体}
%\setlength{\parindent}{2em}
%如果不要缩进 用\noindent
\title{Pinyin Input Method Editor\\Design Report}

\author{\large Yihong Gu\\gyh15@mails.tsinghua.edu.cn\\Department of Computer Science\\Tsinghua University}

\date{}

\begin{document}

\maketitle

\section{Introction}

报告分为三个部分:

\begin{itemize}
	\item Language Model: 介绍所用的语言模型
	\item Search Algorithm: 介绍所用的搜索算法以及优化
	\item Experiments: 给出实验结果并作相关分析
	\item Conclusion and Furthor Work: 总结和指出可以改进的地方
\end{itemize}

\section{Language Model}

\subsection{Probability Model}

总体来说,我们使用以下语言模型:

\begin{eqnarray}
	\mathbb{P}(w_1\cdots w_n) = \prod_{i=1}^{\min(n,m)}{\mathbb{P}(w_i\lvert w_{\max(i-m+1,1)} \cdots w_{i-1})}
\end{eqnarray}

我们把这个模型称为m-gram模型。

在这里面$w_i$表示第$i$个汉字,举个例子,取$m=2$:

\begin{eqnarray}
	\mathbb{P}(\text{清华大学})=\mathbb{P}(\text{清})\mathbb{P}(\text{华}\lvert\text{清})\mathbb{P}(\text{大}\lvert\text{华})\mathbb{P}(\text{学}\lvert\text{大})
\end{eqnarray}

事实上,这里面我们没有考虑拼音的影响,那么,我们作最简单的假设,假设拼音和$m-gram$独立并且条件分布是离散分布

\begin{eqnarray}
	\mathbb{P}(w_1\cdots w_n\lvert t_1 \cdots t_n) = \prod_{i=1}^{\min(n,m)}{\mathbb{P}(w_i\lvert w_{\max(i-m+1,1)} \cdots w_i-1)\mathbb{P}(w_i \lvert t_i)}
\end{eqnarray}

我们让

\begin{eqnarray}
	\mathbb{P}(w_i\lvert w_{i-m+1}\cdots w_{i-1}) = \frac{\#\{w_{i-m+1}\cdots w_{m}\}}{\#\{w_{i-m+1}\cdots w_{m-1}\}}
\end{eqnarray}

其中$\#\{w_{i-m+1}\cdots w_{i}\}$为词组$w_{i-m+1}\cdots w_{i}$在corpus中出现的频数,并且让$\mathbb{P}(w_i \lvert t_i)$为$1$当且仅当汉字$w_i$存在发音$t_i$,否则为$0$,我们也尝试了其他的模型(均匀分布,按汉字的词频归一化的离散分布,但是发现实际上这些方法会引入大量噪声,实际效果并没有之前这种简单也不归一化的方法好,因为前一种方法让文本完全由corpus决定,不引入拼音造成的噪声)。

\subsection{Frequency Count}

在计算$\#\{w_{i-m+1}\cdots w_{i}\}$的过程中,我们使用sina新闻2016作为corpus,且把所有的6763个汉字作为$w_i$的字母表$\Sigma$,把新闻正文中不属于$\Sigma$的部分作为分隔符,统计在$\Sigma$中的连续$m$个token(中间不能有分隔符)出现的次数。

由于总的次数过于多,我们考虑只保留部分m-gram的频数统计的结果,我们选取最大的$k$,使得频数$\ge k$的m-gram的频数之和大于总频数之和的$100 \sigma\%$,我们把$\sigma$称为significance level,在这里我们取$\sigma=0.95$,最后我们保留频数$\le k$的m-gram。

\subsection{Probability Smoothing}

首先,为了方便计算,我们同意使用概率取对数进行计算,这样原来的乘积就变成了求和。

由于词频很多时候都为0,所以我们需要用对$\log\mathbb{P}(w_i\lvert w_{i-m+1}\cdots w_{i-1})$进行平滑处理。

我们下面考虑具体的处理过程(递归处理):

\begin{itemize}
	\item 如果当前发现$w_{i-m+1}\cdots w_{i}$和$w_{i-m+1}\cdots w_{i-1}$的频数均非0,那么就直接按照原式计算$\log\mathbb{P}(w_i\lvert w_{i-m+1}\cdots w_{i-1})$。
	\item 如果发现$w_{i-m+1}\cdots w_{i-1}$的频数均为0,并且$m>2$,计算$m'=m-1$的结果$p_{m-1}$,然后输出就是$p_{m-1}-100$,作为平滑处理的惩罚项。
	\item 如果发现$w_{i-m+1}\cdots w_{i-1}$,并且$m=2$,计算$m'=m-1$的结果$p_{m-1}$,然后输出就是$p_{m-1}-2\times 10^8$,作为平滑处理的惩罚项。
\end{itemize}

另外,由于我们是(要通过搜索)需要最大化对数似然值,所以我们设置答案的下界为$-1\times 10^9$,也就是说,像第三项的那种平滑处理不能超过$5$次。

\section{Search Algorithm}

有了Langugae Model后,我们的问题就转变成了最大化

\begin{eqnarray}
	w_1^*\cdots w_n^* = \mathrm{argmax}_{w_1\cdots, w_n} \mathbb{P}(w_1\cdots w_n\lvert t_1 \cdots t_n)
\end{eqnarray}

其中$t_1 \cdots t_n$是给定的拼音,同时$w_1^*\cdots w_n^*$就是我们输出的结果。

我们考虑使用$A^*$算法来解决这个问题。

\subsection{$A^*$ Algorithm}

我们把$w_1\cdots w_i$称为一个状态$s_i$,当$i=n$的时候即到达终点,一个状态$s_i$的收益为$v_i = \log \mathbb{P}(w_1\cdots w_i)$,我们需要最大化到达终点的收益$v_n$。

服从$A^*$的记号,我们发现$g(s_i)=v_i$,另外我们让$h(s_i)=0$,即可用$A^*$来优化。此时我们发现,这个问题实质上变成了一个最长路径问题,这时候的$A^*$也就等价于传统的Dijkstra算法。

\subsection{Improvement}

我们从以下一个角度来优化这个搜索过程:

\noindent \textbf{SLF优化}

我们发现,是否对OPEN表排序(即使用堆来维护OPEN表)不影响时间消耗,所以我们不对OPEN表排序,这样的搜索算法就等价于传统的SPFA算法,我们沿用了SPFA算法的一个非常经典的优化手法$SLF$优化,即如果放入队尾的状态比放入目前队头的要优的话,把队头队尾的元素交换,这样可以使得效率提升3倍。

\noindent \textbf{记忆化}

我们发现,计算local log probability ($\log\mathbb{P}(w_i\lvert w_{i-m+1}\cdots w_{i-1})$)非常消耗时间,所以我们对这一部分进行记忆化,这样效率也可以提升1倍。

\section{Experiments}

\subsection{Toy data set - Sina News}

我们随机选取了sina新闻(2017.4.9)的四篇不同文章的11个短语/句子,文章列表如下:

\begin{itemize}
	\item 政府工作报告7次提及 李克强为何再赠4字?
	\item 武汉最懒大学生:两周不收衣服 鸟儿在内做窝
	\item 特朗普称叙化武袭击事件是“对人类的羞辱”
	\item 郎平:女排备战奥运会培养新人 已着眼下个周期
\end{itemize}

2-gram的结果如下:

\begin{itemize}
	\item 振兴实体经济是当前一个重要命题/振兴市体经济适当前一个重要命题
	\item 这方面的成功事例数不胜数/这方面的成功实力输部省属
	\item 很明显表达出两层意思/很明显表达出两个意思
	\item 他每天上完晚自习后要去健身房健身/他每天上万万字西后要去健身房间参
	\item 男大学生们普遍表示理解/南大学生们普遍表示理解
	\item 美国总统特朗普在记者会上讲话/美国总统特朗普在记者会上讲话
	\item 发生在叙利亚的针对无辜平民的化武袭击事件/发生在叙利亚的针对无辜平民的话务系及时间
	\item 自己将开始独立制订和执行球队的训练计划/自己将开始都理制定和支行求对的续联系化
	\item 中国队会继续培养新人/中国队会继续培养心人
	\item 最为引人注目的是中国影片/最为引人瞩目的是中国影片
	\item 遵照国际惯例和规则/遵照国际管理和规则
\end{itemize}

最后准确率为$76\%$,我们发现,他很难刻画一些长词/长句,比如"国际惯例"、"数不胜数"、"是当前"、“上完晚自习”,他会把这些长词变成一些二字词语接龙,比如"数不胜数"变成了"输部"+"部省"+"省属"。

3-gram的结果如下:

\begin{itemize}
	\item 振兴实体经济是当前一个重要命题/振兴实体经济是当前一个重要命题
	\item 这方面的成功事例数不胜数/这方面的成功实力数不胜数
	\item 很明显表达出两层意思/很明显表达出两个疑似
	\item 他每天上完晚自习后要去健身房健身/他每天上完晚自习后要去健身房间参
	\item 男大学生们普遍表示理解/南大学生们普遍表示理解
	\item 美国总统特朗普在记者会上讲话/美国总统特朗普在记者会上讲话
	\item 发生在叙利亚的针对无辜平民的化武袭击事件/发生在叙利亚的针对无辜平民的化物袭击事件
	\item 自己将开始独立制订和执行球队的训练计划/子即将开始都理制定和执行求对的续联系华
	\item 中国队会继续培养新人/中国队会继续培养新人
	\item 最为引人注目的是中国影片/最为引人注目的是中国影片
	\item 遵照国际惯例和规则/遵照国际惯例和规则
\end{itemize}

最后准确率为$86\%$,我们发现,他已经能够刻画一些四字词语,比如"国际惯例"、"数不胜数"、"是当前"、“上完晚自习”。这是非常值得表扬的。

4-gram的结果类似3-gram,最后的准确率为$85\%$,没有明显的提升。

\subsection{Toy data set - math}

我们选取了夏道行的《实变函数与泛函分析<上>》中的10个短语,作为第二个toy data set,查看具体的效果,这里我们展示n-gram为3的效果。

\begin{itemize}
	\item 虽然已经解决了建立新积分方法的首要问题/虽然已经解决了坚力新计分方法的首要问题
	\item 建立了较一般集上的测度理论/建立了较一般计上的策都理论
	\item 后面我们将称具有这种性质的函数为可测函数/[Can't Found Answer]
	\item 下面引入可测函数的概念/下面引入可测函数的概念
	\item 可测函数的有限可加性/可测汉书的有限可嘉兴
	\item 几乎处处收敛函数列的控制收敛定理/[Can't Found Answer]
	\item 证明积分与极限交换顺序/证明其分与其见交还顺序
	\item 再举一些控制收敛定理的应用/载具一些空置受联鼎立的影用
	\item 读者自己也可以列举并加以证明/读者自己也可以列车并加以证明s
	\item 所以这两个函数几乎处处相等/所以这两个寒暑期护处处相等
\end{itemize}

最后准确率为$53\%$。值得欣喜的是,即使语料库中没有"可测函数"这个词语,他最后也能打出来,这与我们的平滑处理中的$2\times10^8$的那一项密切相关。同时,我们可以发现,即使我们选取的都是数学书中比较贴近生活用语的句子,他的表现也不是非常好,比如"列举并加以证明"、“函数几乎”这些分开来说得通的词语合起来却无法打出。

\subsection{Overall Test Set Performance}

我们考虑在整个测试集上的表现,几个ngram的表现分别如下:

\begin{itemize}
	\item 2-gram:
	\item 3-gram: $73\%$
	\item 4-gram: 
\end{itemize}

\subsection{Samples}

\noindent \textbf{Well Done Samples}

我们具体分析几个例子:

\begin{itemize}
	\item 对染色体人工合成的工作给予了高度评价
\end{itemize}

我们查看其local log probability:

\begin{itemize}
    \item 对: 14.224
    \item 染: -200000000.000
    \item 色: -105.043
    \item 体: -100.540
    \item 人: -205.390
    \item 工: -205.242
    \item 合: -104.873
    \item 成: 0.000
    \item 的: -1.108
    \item 工: -104.803
    \item 作: -0.246
    \item 给: -5.245
    \item 予: -0.063
    \item 了: -1.134
    \item 高: -1.874
    \item 度: -0.018
    \item 评: -0.908
    \item 价: -0.002
\end{itemize}

我们发现,实际上没有“对染”这个2-gram,所以我们引入了smoothing中的$2\times10^8$,让他能够断词,但是需要付出巨大代价(能不断就不断),同时注意这里的对的local log likelihood是正的这一点是为了方便计算,是直接对频数取log的结果,他和对频率取log之差一个常数,所以忽略了这个常数。

下面也是几个“出乎意料”的比较好的结果:

\begin{itemize}
	\item 美女与野兽
	\item 深度神经网络对计算资源的消耗很大
	\item 北京的房价是否在透支年轻人的创造力
	\item 人文和工业工程必将会师决赛
	\item 人与人之间为什么要互相伤害呢
\end{itemize}

他们的句式都不是偏新闻的句式,但是效果都还不错。

\noindent \textbf{Poor Done Samples}

再看几个做得不好的例子:

\begin{itemize}
	\item 拟(你)的世界会变得更精彩
	\item 请大家选择你觉得可疑(可以)的时间
	\item 读者自己也可以列车(举)并加以证明
	\item 现同期(先统计)大量真实与了(语料)中各个词出现的概率
	\item 我从未见过有如此后演舞池(厚颜无耻)之人
\end{itemize}

(1) 这主要是由于“$P(\text{世}\lvert\text{拟的}>P(\text{世}\lvert\text{你的}$”,因为“你的”这样的太常见了,所以前面是“你的”后面接“世”的概率就会比较小。

(2) 这是由于corpus的保留的不合理导致的,实际原因就是“疑的时”保留在language model中但是“以的时”没有,实际上这两者都不应该被保留

(3) 列车(举):是由于多音字的混乱引起的,拼音为lie ju,车有ju的音,但是在文本中却是列车(che),同时列车这个文本比列举出现得更多。

(4) 语料库不丰富导致的问题。

(5) 断词的问题,没有“此厚”和“耻之”。

同时,我们也注意到了这样的模型的不稳定性,看下面的例子:

\begin{itemize}
\item 人与人之间为什么要互相上海阿(伤害啊)
\item 人与人之间为什么要互相伤害呢
\item 人与人之间为啥要互相伤害呢
\item 仁与人之间怎么就不能互相伤害呢
\item 人与人之间就是要和向上海阿(互相伤害啊)
\end{itemize}

类似的语句有的却输出不是很好的结果。

\section{Conclusion and Furthor Work}

我们可以发现,这个框架的好坏基本上是由language model的好坏来决定的,事实上现在的这个language model不是一个很好的模型,关键在于现在的模型是由字决定的,这就会导致之前的一系列的问题。同样,现在的corpus的局限性也比较大,我们可以在之后考虑几个改进的地方

\begin{itemize}
	\item 考虑更广泛的语言模型:可以让拼音起到一定作用解决多音字的影响,把词作为基本单位。
	\item 考虑使用更加好的corpus,事实上,wiki或者baike的效果应该会比新闻要好一点(从英文的word2vec可以看出)。
\end{itemize}
\end{document}